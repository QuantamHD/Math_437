\documentclass{article}
  \usepackage{amssymb}
  \usepackage{geometry}
  \usepackage{amsmath}
  \usepackage{placeins}


  \begin{document}
  
  \title{Chapter 8}
  \author{Ethan Mahintorabi}
  
  \maketitle

  \section*{8.B}
    \subsection*{(i)}
      \paragraph{} Consider $a = 87$ and $b = 8$. We have $S = \{ x \in \mathbb{Z} | 0 \leq 8x \leq 87\} = \{0,1,2,3,4,5,6,7,8,9,10\}$. So $maxS = 10 =q$ and $r = 87 -  8 \cdot 10 = 7 = r $

      \paragraph{} Consider $a = 138$ and $b = 17$. We have $S = \{ x \in \mathbb{Z} | 0 \leq 17x \leq 87\} = \{0,1,2,3,4,5,6,7,8\}$. So $maxS = 10 =q$ and $r = 138 -  17 \cdot 8 = 2 = r $


      \paragraph{} Consider $a = 192$ and $b = 12$. We have $S = \{ x \in \mathbb{Z} | 0 \leq 12x \leq 87\} = \{0,1,2,3,4,5,6,7,8,9,10,11,12,13,14,15,16\}$. So $maxS = 16 =q$ and $r = 192 -  12 \cdot 16 = 0 = r $

    \subsection*{(ii)}
      \paragraph{Example 1}We will find the inverse $[3]$ in $\mathbb{Z}/7\mathbb{Z}$ starting with the division algorithm and the expression

      \[
        \begin{split}
          7 &= 3 \cdot 2 + 1.\\
        \end{split}
      \]
      
      \noindent Now by solving for the remainder we have 

      \[
        \begin{split}
          1 &= 7 \cdot 1  + 3 \cdot -2.\\
        \end{split}
      \]

      \noindent Lastly substituting our expressions in reverse we see that the multiplicative inverse is equal to $[-2]$ or $[5]$.


      \paragraph{Example 2} We will find the inverse $[5]$ in $\mathbb{Z}/19\mathbb{Z}$ starting with the division algorithm and the expression

      \[
        \begin{split}
          19 &= 5 \cdot 3 + 4\\
          5 &= 4 \cdot 1 + 1.\\
        \end{split}
      \]
      
      \noindent Now by solving for the remainder we have 

      \[
        \begin{split}
          1 &=  5  - 4 \cdot 1\\
          4 &=  19 - 5 \cdot 3.\\ 
        \end{split}
      \]

      \noindent Lastly we will apply the definitions in reverse to solve bezot's identity for

      \[
        \begin{split}
          1 &=  5  - (19 - 5 \cdot 3) \cdot 1\\
            &=  5 \cdot 4 - 19\\
            &= 5 \cdot (19 - 5 \cdot 3) - 19\\
            &= 19 \cdot 4  + 5 \cdot -15.
        \end{split}
      \]

      \noindent As we can see the multiplicative inverse of $[5]$ in $\mathbb{Z}/19\mathbb{Z}$ is $[-15] = [4]$.


      \paragraph{Example 3} We will find the inverse $[17]$ in $\mathbb{Z}/37\mathbb{Z}$ starting with the division algorithm and the expression

      \[
        \begin{split}
          37 &= 17 \cdot 2 + 3\\
          17 &= 3 \cdot 5 + 2\\
          3 &= 2 \cdot 1 + 1
        \end{split}
      \]
      
      \noindent Now by solving for the remainder we have 

      \[
        \begin{split}
          1 &=  3  - 2 \cdot 1\\
          2 &=  17 - 3 \cdot 5\\
          3 &= 37 - 17 \cdot 2 
        \end{split}
      \]

      \noindent Lastly we will apply the definitions in reverse to solve bezot's identity for

      \[
        \begin{split}
          1 &=  3  - (17 - 3 \cdot 5) \cdot 1\\
          &= 3 \cdot 6 - 17\\
          &= 6 \cdot (37 - 17 \cdot 2) - 17 \\
          &= 6 \cdot 37 - 12 \cdot 17 - 17\\
          &= 6 \cdot 37 + -13 \cdot 17
        \end{split}
      \]

      \noindent As we can see the multiplicative inverse of $[17]$ in $\mathbb{Z}/37\mathbb{Z}$ is $[-13] = [24]$.


\end{document}