\documentclass{article}
  \usepackage{geometry}
  

  \begin{document}
  
  \title{Reflection 4}
  \author{Ethan Mahintorabi}
  \date{February 28, 2018}
  
  \maketitle

  
  \section*{Article}
    \paragraph{}
    Believing that ones abilities are limited by natural born talent necessarily implies that effort is worthless as we can never move beyond our current limitations. Dweck labels this philosophy the "fixed mindset". Where as the "growth mindset" posits that when we accept that we can grow beyond what we are today it allows people to accept their current limitation and move pass them. These two mindsets can explain why some ``smart'' students succeed at first then fail as they advance in academia while others succeed. Dweck's article explains that with a fixed mindset students are more likely to give up in the face of increasing difficulty because they falsely believe that their struggle is due to lack of natural ability. 
    
    \paragraph{}
    If your fundamental assumption growth is fixed, then it is only natural to become defensive when your performance suggests is less than perfect because under the fixed mindset assumption that deficiency cannot be remedied. The growth mindset helps combat those feelings of inadequacy by explaining to students that their abilities are not fixed and they can improve. Dweck also points out that we should praise children for their methods and perseverance rather than their traits like intelligence or athletic ability to foster a healthy growth mindset.

  \section*{Videos}
    \paragraph{}
    The discussion between Khan and Dweck highlights the importance of a growth mindset in education. Dweck explains that her research shows that students with a growth mindset earn better grades than their cohorts who do not. Dweck's research also shows that if we can teach students that they are not limited by natural ability they will be more willing to take on challenges head on inside and outside of the classroom.
        

  \section*{Math Reflection}
    \paragraph{}
    When I was 4 years old, I was sure that I wanted to be a doctor specifically a cardiac surgeon. Adults would tell me that you need to be really smart to be a doctor not just anyone could do it. As far as I can remember I would usually respond by saying, ``anyone can be a doctor if they really tried''. I would have similar arguments with my mother about whether someone could learn to draw. I was always one to believe that anyone could do anything if they wanted to, but most people didn't want to.

    \paragraph{}
    During my early college years my opinion changed. Other students would tell me they were not excellent at math, science, or interpretive tuba and I started to wonder why that was. I questioned whether natural born talent really die play a role in academic success. At that time I wasn't ready to admit that talent alone was the only deciding factor in success, but I was starting to question my previous convictions. The way I bridged these two conflicting ideals was to accept that effort was the key factor to success, but that some people are naturally drawn to a subject which makes it easier to put in that effort.

    \paragraph{}
    Today I believe that there is a limitless possibility for growth in any subject field or activity. I still believe that some are more drawn to certain subjects. Now that I'm aware of the growth mindset it is much easier to apply it to real world situations like struggling through linear algebra.
    
  \end{document}