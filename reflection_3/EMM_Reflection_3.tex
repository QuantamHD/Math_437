\documentclass{article}
  \usepackage{geometry}
  

  \begin{document}
  
  \title{Reflection 3}
  \author{Ethan Mahintorabi}
  
  \maketitle

  
  \section*{Book Chapter}
    \paragraph{}
         The book presented two compelling techniques to aide  in solving problems. the first technique, analyzing a problem by looking at its extremes, I was familiar with and have applied the technique successfully. The extrema technique is a common technique to solve computer science questions because by often coding for the edge cases the behavior of an algorithm in the normal case appears as part of the extrema checks. I had not seen the technique formalized to a set of steps before. However, now that I am aware of the formal technique; I can apply more directly when I approach problem. The less familiar technique from the book presented a technique wherein the user intentionally solves the problem incorrectly and then identifies the problems with the proposed arguments solving each incorrect component one at a time. Once there are no i more incorrect arguments to fix you have solved the problem. I can see how directly applying this technique could very well help students solve difficult math questions.

  \section*{Videos}
    \paragraph{}
        The two focus on the idea that failure is more valueable than success, and beyond it being more useful than success it is also a necessary precursor to it. The Micheal Jordan video reads off one of his famous quotes about all the ways he has ``failed'' his team in the past, but that the only reason he can succeed is because those failures have shown him all the ways he can improve his game and the only reason he can succeed at all. The TedX talk also brought up a similar point that to identify what is good we must first fail and then we learn what methods are actually effective from our failed attempts. The key point of both videos is that one cannot succeed without failures and that it is both good to fail and a necessary component of success.
        

  \section*{Math Reflection}
    \paragraph{}
        The  second book chapter, assigned for reading, resonated with a lot of my math expirences in the past particularly the part about not using your mistakes to advance your understanding of a subject is detrimental to a students understanding of the material taught. Specifically in Calculus III I never understood the concept of Lagrange multipliers and the concept appears in many of the optimization applications for example in quadratic programming. Whenever I read about those subjects, I feel as though that I have an unstable foundation to learn this material. The mentions that one possible side effect of learning 80 percent of the material is that my foundations going forward would be weak.I think that by forcing my self to learn from my mistakes when given the opportunity I'll be able to repair my gaps in knowledge going forward and prevent new ones from forming.


    
  \end{document}