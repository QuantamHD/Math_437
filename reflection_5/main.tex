\documentclass{article}
  \usepackage{geometry}
  

  \begin{document}
  
  \title{Reflection 5}
  \author{Ethan Mahintorabi}
  
  \maketitle

  
  \section*{Problem 6.A.vi}
    When I approached this problem I noticed that I needed to prove an equivalence relationship and it looked like a difficult one to prove. I didn't know where to start and I was in a way afraid of the problem, but I knew that I needed to solve it. So after I calmed down a bit I looked at the problem and really analyzed what it was saying; I thought to myself ``okay so all I need to do is prove this equivalence relationship''. So, I looked up the three properties of an equivalence relationship that you need to prove namely the symmetric, reflexive and transitive properties. I looked at what I could assume at each step and how that related to rings and ideals. The first one was easy, ``$x-x$ is always zero'' I thought; that must always be in an ideal, and it felt great tackling the first step. As I moved through the problem analyzing each step and by analyzing the properties could be used from the rings and ideals was able to prove the theorem. What got me unstuck was seeing exactly what needed to be proven in the large theorem and tackling the steps one by one.

  \section*{Problem 6.B.iii}
    Often, when doing my home work I look ahead at future problems to get an idea of what I might need to accomplish. That's when I noticed this problem, as I was on Problem 6.B.i I didn't know how to approach it, and couldn't think of a theorem to use to start. I felt unconfidant that I would be able to solve it; so I decided to move on from problem 1 to problem 2 and that's when it hit me. If the second problem was true I knew exactly how I could approach the third problem. Solving the second problem gave me an approach to solve the third by using the fact that fields only have two ideals zero and themselves. Once I had solved the previous problem the third one seemed like a simple corollary in comparison. 

    
    
  \end{document}