\documentclass{article}
  \usepackage{geometry}
  \usepackage{amssymb}
  

  \begin{document}
  
  \title{Reflection 6}
  \author{Ethan Mahintorabi}
  
  \maketitle

  
  \section*{Problem 6.B.iii}
    \paragraph{} This problem dealt with the relationship between the four sets $\mathbb{Z}, \mathbb{Q}, \mathbb{R}$, and $\mathbb{C}$ and whether or not any of those sets are ideals of each other. I saw this problem while I was working problem (i) in the same section, and this one popped out at me because I wasn't quite sure how to approach it and instead of attacking each problem in order I decided to work on the third before I finished the first and second.

    \paragraph{} This had the unintended consequence of missing out on a key piece of information in the second problem. Without that key piece of information I tried to approach it from the definition of an ideal and show that the integers could not be an ideal of the rationals because there are some integers when multiplied by another rational number would still be a rational. Something about this argument seemed incorrect to me because it felt weak especially in the case of the integers as an ideal of the reals because there is no set algebraic form of the reals. I thought I could show that $\pi \cdot 1$ would result in $\pi$ which was not in the integers, but once again it felt incorrect because I would need to show that $\pi$ is not an integer or rational to complete the problem.
    
    \paragraph{} Once I was sufficiently stuck I stopped working on this problem, and then moved onto solving problem i and then ii. Once I got to question ii; I realized that this problem was the key to solving iii because $\mathbb{Q}, \mathbb{R}$, and $\mathbb{C}$ would make quick work of the next problem. The first failed attempt made me mindful of the techniques that were unsuccessful and that a new approach was needed. That openness lead me to quickly to the technique I was looking for once I saw problem ii. The overarching lesson of this productive failure is that there is sometimes a reason problems are ordered to teach you lessons in a certain order and skipping ahead can ruin that continuity.

  \section*{Problem 7.A.ii}
    This problem took me over three days to solve, and numerous failures before I finally understood one how to approach the problem, and two understand it's relationship to the complex numbers. In this problem we were tasked with explaining the reason that the coset representatives were of the form $a + bX$. I first attempted to use the equivalence relationship of the form $\{s\in R | s-r = a\in I\}$, and then select a general form for $s = (a + a_1X + \dots + a_nX^n$) and $a= (X^2 + 1)(a + a_1X + \dots + a_nX^n)$ then solve for $r$. However, I couldn't show that it was of the form $a + bX$, but the technique lead me to use the $\{s\in R | s = r + a\in I\}$ definition instead of the difference definition. After I realized there was no element in the ideal $X^2 + 1$ of degree less than two and to satisfy the $\{s\in R | s = r + a\in I\}$ definition the coset representatives must be of the form $a + bX$.

    \paragraph{} The general lesson  here is sometimes it is more useful to look at the structure of objects instead of trying to manipulate their algebraic forms directly. 

    
  \end{document}