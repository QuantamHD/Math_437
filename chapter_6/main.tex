\documentclass{article}
  \usepackage{amssymb}
  \usepackage{geometry}
  \usepackage{amsmath}
  \usepackage{placeins}


  \begin{document}
  
  \title{Chapter 6}
  \author{Ethan Mahintorabi}
  \date{April 12, 2018}
  
  \maketitle

  \section*{(6.A)}
    \subsection*{(i)}
      We know that both $(0)$ and $R$ are subrings of every every ring $R$ by exercise (ii) in 5.A. We will first verify that the set $R$ is in fact an ideal; we know that for any two elements $a,b \in R$ that $ab$ will always be in $R$ by the definition of a ring and is also the case that $ba \in R$ and because it is a subring and holds $ab, ba \in R$ by \textbf{Theorem 6.5} we know it is an ideal. In the case of $(0)$ the only element in the ring is $0$ and for any element $a \in \{0\}$ and any element $b \in R$ we know that $ab = ba = 0$ by 2.D implies that by \textbf{Theorem 6.5} is an ideal of $R$.

    \subsection*{(ii)}
      If we have a sequence of elements $(a_1, a_2, \dots, a_n)$ generating an ideal, we know that every element will be of the form $c = \sum_{i}^{i=1} a_ib_i$ where $c \in I$ and $b_i \in R$. This is true because if we think about the smallest ideal in the case of $n=1$ or $(a_1)$ by the definition of an ideal the set of elements would be of the form $\{a_1b | b \in R\}$. We can extend this logic to the case where we have two elements $(a_1, a_2)$ we know that both $\{a_1b_1 | b_1 \in R\}$ and $\{a_2b_2 | b_2 \in R\}$ by the definition ideals, but we also know that their sum must also be in the ideal by the definition of a ring. Thus, in general the form of an element of an ideal generated by a set will be the sum of the terms multiplied by some element in the ring.

    \subsection*{(iii)}
      To show that the left ideal of $M_2(\mathbb{R})$ generated by $\begin{bmatrix}1 & 0 \\ 0 & 0 \end{bmatrix}$ is the set of matrices in $M_2(\mathbb{R})$ of the form $\begin{bmatrix}a & 0 \\ c & 0 \end{bmatrix}$ where $a,c \in \mathbb{R}$ we will first show that every element of the form $\begin{bmatrix}a & 0 \\ c & 0 \end{bmatrix}$ is in the left ideal generated by $\begin{bmatrix}1 & 0 \\ 0 & 0 \end{bmatrix}$.
        
      \noindent
      \paragraph{} We will take an element $r \in M_2(\mathbb{R})$ which is of the form $\begin{bmatrix}a & b \\ c & d \end{bmatrix}$ where $a,b,c,d \in \mathbb{R}$ and the element $\begin{bmatrix}1 & 0 \\ 0 & 0 \end{bmatrix}$ which will generate our ideal and will multiply these two elements $\begin{bmatrix}a & b \\ c & d \end{bmatrix}\begin{bmatrix}1 & 0 \\ 0 & 0 \end{bmatrix} = \begin{bmatrix}a & 0 \\ c & 0 \end{bmatrix}$. Since the element $\begin{bmatrix}1 & 0 \\ 0 & 0 \end{bmatrix}$ will always generate an element of the form $\begin{bmatrix}a & 0 \\ c & 0 \end{bmatrix}$ we have shown that every element in the ideal generated by $\begin{bmatrix}1 & 0 \\ 0 & 0 \end{bmatrix}$ is of the form $\begin{bmatrix}a & 0 \\ c & 0 \end{bmatrix}$. Now we must show that the ideal generated by $\begin{bmatrix}1 & 0 \\ 0 & 0 \end{bmatrix}$ is in fact an ideal.

      \paragraph{} We will begin by showing that matrices of the form $\begin{bmatrix}a & 0 \\ c & 0 \end{bmatrix}$ are a subring of $M_2(\mathbb{R})$. By the second subring test we need only show that for any $a,b \in I$ the following holds $a-b \in I$ and $ab \in I$ where $I$ is the ideal generated by $\begin{bmatrix}1 & 0 \\ 0 & 0 \end{bmatrix}$. We will take the following elements $a = \begin{bmatrix}a & 0 \\ c & 0 \end{bmatrix}$ and $b = \begin{bmatrix}d & 0 \\ e & 0 \end{bmatrix}$ where $a,c,d,e \in \mathbb{R}$. To show that $ab \in I$ we will multiply

      \[
        \begin{split}
        \begin{bmatrix}a & 0 \\ c & 0 \end{bmatrix} \cdot \begin{bmatrix}d & 0 \\ e & 0 \end{bmatrix} &= \begin{bmatrix}ad & 0 \\ cd & 0 \end{bmatrix}\\
        \end{split},
      \]

      and we find that the resulting matrix is still of the form $\begin{bmatrix}a & 0 \\ c & 0 \end{bmatrix}$ due to the closure properties of the real numbers, and satisfies the condition $ab \in I$. The last property that needs to be shown is that for $a,b \in I$ it is always true that $a-b \in I$. Given the same $a$ and $b$ as defined earlier we can see that
        
        \[
          \begin{split}
          \begin{bmatrix}a & 0 \\ c & 0 \end{bmatrix} - \begin{bmatrix}d & 0 \\ e & 0 \end{bmatrix} &= \begin{bmatrix}a-d & 0 \\ c-e & 0 \end{bmatrix}\\
          \end{split}.
        \]

      Since $a-b$ is still of the form $\begin{bmatrix}a & 0 \\ c & 0 \end{bmatrix}$ due closure under addition over the real numbers we have shown that the matrices of the form $\begin{bmatrix}a & 0 \\ c & 0 \end{bmatrix}$ are a subring. Now we need to show that the set generated by $\begin{bmatrix}1 & 0 \\ 0 & 0 \end{bmatrix}$ hold for the following condition $\begin{bmatrix}e & 0 \\ f & 0 \end{bmatrix} \in I$ and $\begin{bmatrix}a & b \\ c & d \end{bmatrix} \in M_2(\mathbb{R})$ such that $\begin{bmatrix}a & b \\ c & d \end{bmatrix} \cdot \begin{bmatrix}e & 0 \\ f & 0 \end{bmatrix} \in I$. Take the following general matrix multiplication 

        \[
          \begin{split}
          \begin{bmatrix}a & b \\ c & d \end{bmatrix} \cdot \begin{bmatrix}e & 0 \\ f & 0 \end{bmatrix} &= \begin{bmatrix}ae + bf & 0 \\ ce + df & 0 \end{bmatrix}\\
          \end{split},
        \]
      
      since $ae+bf, ce + df \in R$ due to the closure under the reals we know that this is still of the form $\begin{bmatrix}a & 0 \\ c & 0 \end{bmatrix}$ and thus $\begin{bmatrix}a & b \\ c & d \end{bmatrix} \cdot \begin{bmatrix}e & 0 \\ f & 0 \end{bmatrix} \in I$. Thus, we have shown all the properties of a left ideal thus we can conclude that the ideal generated by $\begin{bmatrix}1 & 0 \\ 0 & 0 \end{bmatrix}$ is a left ideal.

    \subsection*{(iv)}
      The ideal $(2, x^2)$ of $\mathbb{Z}[x]$ consists of all polynomials with integer coefficients where the constant and linear terms are even. The reason this is the case is that the ideal generated by $(2, x^2)$ contains all polynomials with even coefficients as a result of the $(2)$ and contains all polynomials where the constant and linear terms are zero due to $(x^2)$. The sums of these two form a set where all the terms of greater degree than $x$ can have any integer coefficients while the linear and constant terms are limited to the even integers.

    \subsection*{(v)}
      The reason that that the containment $(a_1, a_2, \dots, a_n) \subsetneq (a_1, a_2, \dots a_n, a_{n+1})$ does not hold is because there are elements $a_{n+1}$ that add no new elements to the ideal. Take for example the ideal $(2)$ of $\mathbb{Z}$ this ideal contains all of the even integers; that is all integers of the the form $2n$ where $n \in \mathbb{Z}$. If we add another element of $\mathbb{Z}$ to this ideal; say the element $4$ such that we have the ideal $(2,4)$. We know this ideal contains all elements of the form $2n + 4q$ where $n,q \in \mathbb{Z}$. However, this form can be manipulated to show that the expression is still of the form $2(n+2q)$ thus contains the same elements as the smaller sequence. Thus, the containment property must be $(a_1, a_2, \dots, a_n) \subseteq (a_1, a_2, \dots a_n, a_{n+1})$.
      
    \subsection*{(vi)}
      To prove theorem 6.14 we must prove the three properties of an equivalence relationship we will begin with the symmetric property. That is for $a \in R$ we must show that $a - a \in I$, and since $a-a = 0$ for every $a \in R$ we know that $a-a$ must be in $I$ because zero is by definition a part of every ideal.

      \paragraph{} Next we will show that the relation is reflexive; that is if $a-b \in I$ then $b-a \in I$. Because we know that $a-b \in I$ we also know that the additive inverse of $a-b$ must be in $I$ by the definition of a subring. This additive inverse can be calculated as $- (a-b)$ or $b-a$ thus we have shown that $b-a \in I$ and that the relationship is reflexive.

      \paragraph{} The last property that we need to show is that the relationship is transitive that is if $a-b \in I$ and $b-c \in I$ where $a,b,c \in R$ then $a-c \in I$. Under the rules of addition of a subring we know that for any two elements in the subring the sum of the two elements must also be in the ring. Thus, we know that $(a-b) + (b-c) \in I$ which can be reduced to $a-c \in I$ which implies that the relationship is also transitive. Since all three properties hold we have proven that this is an equivalence relationship.

  \section*{6.B}
      \subsection*{(i)}
        In a ring with identity the sequence $(1)$ generates an ideal to form the set $I = \{1a | a \in R\}$ in that case we can see that every element in the set will correspond to the same element in the ring. Thus, we can see that the ideal contains the same elements as $R$ and is $R$.

      \subsection*{(ii)}
        To show that in a field $F$ the only ideals are $(0)$ and $F$ we will first assume there is an ideal $I$ of $F$ which is not $(0)$ or $F$. In this ideal we will take an element $a \in I$ where $a \neq 0$. By the definition of an ideal the element $ab \in I$ where $b \in F$ must hold, and the definition of a field states there is a multiplicative identity for every element other than zero. Thus, we know that $aa^{\prime} \in I$ and as a corollary to that $1 \in I$. Since the previous problem showed that if $1$ is present in an ideal of a ring with identity it becomes the ring itself; it must be the case that this ideal is $F$. We have arrived at a contradiction because we assumed that the ideal I was not $(0)$ or $F$,thus the only ideals of a field are $(0)$ and $F$.

      \subsection*{(iii)}
        In the last problem we found that the only ideals of a field are $(0)$ and $F$, and $\mathbb{Q}, \mathbb{R}, \mathbb{C}$ are fields that means they cannot be ideals of each other because their only ideals are either themselves or $(0)$. In the case of the integers we know that $\mathbb{Q}, \mathbb{R}, \mathbb{C}$ cannot be ideals of $\mathbb{Z}$ because the integers are a proper subset of all three sets and thus they cannot be a subring of $\mathbb{Z}$ and hence not an ideal. Thus, none of these sets are ideals of each other.

      \subsection*{(iv)}
        Since two of the subrings of $M_2(\mathbb{R})$ that I choose were the two trivial ideals $0$ and $M_2(\mathbb{R})$. I would like to choose more informative examples this time. Namely, these three subrings $M_2(\mathbb{Z})$, $M_2(\mathbb{Z}/2\mathbb{Z})$, and $M_2(\mathbb{Q})$.

        \paragraph{} We will begin with the subring $M_2(\mathbb{Q})$ in this ring we only need to verify that this is not an ideal because there are some real numbers when multiplied by rational numbers result in a nonrational number. Take for example

        \[
          \begin{split}
            \begin{bmatrix}2 & 2 \\ 2 & 2\end{bmatrix} \cdot \begin{bmatrix}\pi & \pi \\ \pi & \pi\end{bmatrix} &= \begin{bmatrix}2\pi + 2\pi & 2\pi + 2\pi \\ 2\pi + 2\pi & 2\pi + 2\pi\end{bmatrix}\\
            &= \begin{bmatrix}4\pi & 4\pi \\ 4\pi & 4\pi \end{bmatrix}
          \end{split}
        \]

        \noindent where $\begin{bmatrix}2 & 2 \\ 2 & 2\end{bmatrix} \in M_2(\mathbb{Q})$ and $\begin{bmatrix}\pi & \pi \\ \pi & \pi\end{bmatrix} \in M_2(\mathbb{R})$; in that case we can see that the resulting matrix is not a part of the set $M_2(\mathbb{Q})$ because $4\pi$ is not rational. The counterexample shows that this is not a right ideal. To show that it is not a left ideal we will adopt a similar counterexample given

          \[
            \begin{split}
              \begin{bmatrix}\pi & \pi \\ \pi & \pi\end{bmatrix} \cdot \begin{bmatrix}2 & 2 \\ 2 & 2\end{bmatrix}  &= \begin{bmatrix}2\pi + 2\pi & 2\pi + 2\pi \\ 2\pi + 2\pi & 2\pi + 2\pi\end{bmatrix}\\
              &= \begin{bmatrix}4\pi & 4\pi \\ 4\pi & 4\pi \end{bmatrix}
            \end{split}
          \]

        \noindent where $\begin{bmatrix}2 & 2 \\ 2 & 2\end{bmatrix} \in M_2(\mathbb{Q})$ and $\begin{bmatrix}\pi & \pi \\ \pi & \pi\end{bmatrix} \in M_2(\mathbb{R})$; in that case we can see that the resulting matrix is not a part of the set $M_2(\mathbb{Q})$ because $4\pi$ is not rational. 

        


        \paragraph{} Next the subring $M_2(\mathbb{Z})$ in this ring we only need to verify that this is not an ideal because there are some real numbers when multiplied by an integer that result in a non-integer number. Take for example

        \[
          \begin{split}
            \begin{bmatrix}2 & 2 \\ 2 & 2\end{bmatrix} \cdot \begin{bmatrix}\pi & \pi \\ \pi & \pi\end{bmatrix} &= \begin{bmatrix}2\pi + 2\pi & 2\pi + 2\pi \\ 2\pi + 2\pi & 2\pi + 2\pi\end{bmatrix}\\
            &= \begin{bmatrix}4\pi & 4\pi \\ 4\pi & 4\pi \end{bmatrix}
          \end{split}
        \]

        \noindent where $\begin{bmatrix}2 & 2 \\ 2 & 2\end{bmatrix} \in M_2(\mathbb{Z})$ and $\begin{bmatrix}\pi & \pi \\ \pi & \pi\end{bmatrix} \in M_2(\mathbb{R})$; in that case we can see that the resulting matrix is not a part of the set $M_2(\mathbb{Z})$ because $4\pi$ is not an integer. The counterexample shows that this is not a right ideal. To show that it is not a left ideal we will adopt a similar counterexample given

          \[
            \begin{split}
              \begin{bmatrix}\pi & \pi \\ \pi & \pi\end{bmatrix} \cdot \begin{bmatrix}2 & 2 \\ 2 & 2\end{bmatrix}  &= \begin{bmatrix}2\pi + 2\pi & 2\pi + 2\pi \\ 2\pi + 2\pi & 2\pi + 2\pi\end{bmatrix}\\
              &= \begin{bmatrix}4\pi & 4\pi \\ 4\pi & 4\pi \end{bmatrix}
            \end{split}
          \]

        \noindent where $\begin{bmatrix}2 & 2 \\ 2 & 2\end{bmatrix} \in M_2(\mathbb{Z})$ and $\begin{bmatrix}\pi & \pi \\ \pi & \pi\end{bmatrix} \in M_2(\mathbb{R})$; in that case we can see that the resulting matrix is not a part of the set $M_2(\mathbb{Z})$ because $4\pi$ is not integer.

          
        \paragraph{} Finally, the subring $M_2(\mathbb{Z}/2\mathbb{Z})$ in this ring we only need to verify that this is not an ideal because there are some real numbers when multiplied by an an even integer that result in a non-even integer number. Take for example

        \[
          \begin{split}
            \begin{bmatrix}2 & 2 \\ 2 & 2\end{bmatrix} \cdot \begin{bmatrix}\pi & \pi \\ \pi & \pi\end{bmatrix} &= \begin{bmatrix}2\pi + 2\pi & 2\pi + 2\pi \\ 2\pi + 2\pi & 2\pi + 2\pi\end{bmatrix}\\
            &= \begin{bmatrix}4\pi & 4\pi \\ 4\pi & 4\pi \end{bmatrix}
          \end{split}
        \]

        \noindent where $\begin{bmatrix}2 & 2 \\ 2 & 2\end{bmatrix} \in M_2(\mathbb{Z}/2\mathbb{Z})$ and $\begin{bmatrix}\pi & \pi \\ \pi & \pi\end{bmatrix} \in M_2(\mathbb{R})$; in that case we can see that the resulting matrix is not a part of the set $M_2(\mathbb{Z}/2\mathbb{Z})$ because $4\pi$ is not an even integer. The counterexample shows that this is not a right ideal. To show that it is not a left ideal we will adopt a similar counterexample given

          \[
            \begin{split}
              \begin{bmatrix}\pi & \pi \\ \pi & \pi\end{bmatrix} \cdot \begin{bmatrix}2 & 2 \\ 2 & 2\end{bmatrix}  &= \begin{bmatrix}2\pi + 2\pi & 2\pi + 2\pi \\ 2\pi + 2\pi & 2\pi + 2\pi\end{bmatrix}\\
              &= \begin{bmatrix}4\pi & 4\pi \\ 4\pi & 4\pi \end{bmatrix}
            \end{split}
          \]

        \noindent where $\begin{bmatrix}2 & 2 \\ 2 & 2\end{bmatrix} \in M_2(\mathbb{Z}/2\mathbb{Z})$ and $\begin{bmatrix}\pi & \pi \\ \pi & \pi\end{bmatrix} \in M_2(\mathbb{R})$; in that case we can see that the resulting matrix is not a part of the set $M_2(\mathbb{Z}/2\mathbb{Z}))$ because $4\pi$ is not an even integer.

    
\end{document}