\documentclass{article}
  \usepackage{geometry}
  

  \begin{document}
  
  \title{Reflection 2}
  \author{Ethan Mahintorabi}
  
  \maketitle

  
  \section*{Laursen, et al.}
    \paragraph{}
         After reading the study two key facts about the Inquiry Base Learning Model stood out to me. The first is that interaction and student involvement are key to improving outcomes for college level math students for both high and low achieving students. The second is the students who participate in the IBL program regardless of previous skill do better than their cohorts in non-IBL courses. By allowing students to make their own discoveries it gives them a true confidence that enhances their abilities to learn new material down the road.

  \section*{Videos}
    \paragraph{}
        The three videos shared a similar theme that focused on the same point that failure is the first step towards success. Schultz discusses why failing still hurts and how to move past that feeling of failure to truly succeed and apply those failures to future endeavors. In the first video with Glass says that the mere fact we recognize  that aren't doing well means that we have a certain taste for the creative works we are studying and it that taste that will allow us to improve over time. In Glass' mind failure is just the process by which we improve our skills to match our taste. Schultz also points to the fact that failure is a fundamental part of the learning process and we need to move past the emotional response to failure and see it as a learning opportunity and not something that reflects poorly on our character as a whole.

  \section*{Self Reflection}
    \paragraph{}
        I've always had the opinion that anyone can learn anything as long as they are willing to put in the time and effort. However, all the time, effort, and practice in the world does not preclude failure. There is one math experience in particular that sticks out in relation to the IBL method and productive failures; the I tried to learn calculus in high school on my own.
    \paragraph{}
        During my sophomore year of high school I was taking Algebra II, and I knew that I needed to get an A in the course for both semesters or I would not be able to progress to calculus directly, but rather have to take pre-calculus which was basically a repetition of Algebra II. I'm just going to make the story short though and tell you that it didn't happen I received a B both semesters. I could make all the excuses in the world for why I didn't get an A, but the simple answer was I wasn't trying very hard and for the first time I couldn't just scrape by. Once I was relegated to taking pre-calculus I decided to teach myself calculus without taking a course, and I would take and pass the AP exam by myself.

    \paragraph{}
        I wasn't successful in my endeavor the AP exam came and went, and I ultimately got a 1 the worst score you can get. I didn't view it as a failure though just another step in my learning process. In my senior year I took a real calculus course and through my own self study practices I realized that I had learned some calculus during my self study course, and it helped me during the course. With the aide of my self study knowledge I breezed through the course and on my second attempt at  taking the AP exam I received a 5. This story is intimately related with the productive failure model described in the IBL approach. I didn't let my failure discourage me, and it ended up being instrumental to my success down the road. I feel like failure is not the end of the world because no matter how badly you screw up you always learn something new.

    
  \end{document}