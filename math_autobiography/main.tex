\documentclass{article}
  \usepackage{geometry}
  

  \begin{document}
  
  \title{Math Autobiography}
  \author{Ethan Mahintorabi}
  
  \maketitle

  
  \section*{Early Years}
    \paragraph{}
        If there is going to be any reflection on math and my experience with it; I thought the best idea would be to start as early as I can remember. I was in the second grade learning the multiples of numbers up to 10 and as a class we would go through the numbers 1 through 10 and memorize what $1 \cdot 2$ was, $3 \cdot 4$, and so on and so forth. I distinctly remember being told that if I was not able to ``learn'' these multiples I would not be allowed to move on to the 3rd grade. I think it was at this point in my life I thought math was nothing more than just a wrought memorization test. That's how it stayed for me until middle school.

  \section*{Middle School}
    \paragraph{}
      During my time in middle school I was introduced to the topics of Algebra and simple linear equations. This was the first time in my life that I can remember thinking ``Oh, this might actually be interesting''. I passed the first pre-Algebra course with flying colors, and joined an Advanced Algebra course in 8th grade. This was also the first time I felt that a class was kicking my ass, and not because I didn't understand the material, but because I didn't have the discipline yet to keep up with a course that didn't have deadlines. I got a C in that class and I think for the first time in my life I was discouraged and wasn't sure that I could really do well in Math. It wouldn't be until highschool that I really found a place for math in my heart.
  
  \section*{High School}
    \paragraph{}
      In highschool I had a very positive experience with math and the math fields I was introduced to. The first class I took in high school was a simple proofs based geometry class. While we weren't proving anything groundbreaking it was fun to show that two lines were parallel by combining the properties of lines, circle and rectangles. I viewed math as more of a puzzle at this stage in my math careerer. It was that class that restored my confidence that I could be great at math I just needed to put in the time and dedication that it required. 

    \paragraph{}
      High school is also when I became more interested in the more advanced math fields. Specifically in the areas of numerical analysis and how it related to computational physics simulations in the areas of light transport. I was hopeless unprepared to understand it at that time, but it was never about fully understanding the material in reality it was my goal, but never the target. What I wanted was to gain new skills by reaching higher than I was capable of, and building a foundation under myself so that one day I could understand it. It was these sisyphean tasks that made me fall in love with the subject, and how I learn new math to this day; I just pick up a book and read.
  

    
  \end{document}