\documentclass{article}
  \usepackage{amssymb}
  \usepackage{geometry}
  \usepackage{amsmath}
  \usepackage{placeins}


  \begin{document}
  
  \title{Chapter 4}
  \author{Ethan Mahintorabi}
  
  \maketitle

  \section*{(4.A)}
    \subsection*{(i)}
    We will assume that there is an $a \cdot b = 0$ such that $a,b \in R$ and that $b \neq 0$ as is the definition of zero divisor. We then show that

    \begin{equation*}
      \begin{split}
        x \cdot y &= 0\\
        x^{\prime} \cdot x \cdot y &= 0\\
        1 \cdot y &= 0\\
        y &= 0
      \end{split}  
    \end{equation*}

    thus we have a arrived a contradiction because we said that $y \neq 0$.


    \subsection{(ii)}
    We will show that 

    \[
      A = \begin{bmatrix}
        1 & 2\\
        3 & 4
      \end{bmatrix}
    \]

    is a unit matrix by showing that there exists some $A^{\prime} \in R$ such that
    $A^{\prime} \cdot A = 1$. Since we can show that


    \begin{equation*}
      \begin{split}
        A^{\prime} \cdot A &= 1\\
        \begin{bmatrix}
          -2 & 1\\
          \frac{3}{2} & \frac{-1}{2}
        \end{bmatrix} \cdot A \cdot y &= 1\\
      \end{split}  
    \end{equation*}

    In cannot be a zerodivisor by our theorem in (i).


    \subsection{(iii)}
    We will begin by analyzing the nilpotent elements of $\mathbb{Z}/8\mathbb{Z}$
    
\end{document}